\documentclass[12pt]{article}
\usepackage[utf8]{inputenc}
\usepackage{amsmath}
\usepackage{amsthm}
\usepackage{amssymb}
\usepackage{float}
\usepackage[algoruled,linesnumbered,noend]{algorithm2e}
\usepackage{relsize}
\usepackage{enumitem}
\usepackage{cite}
\usepackage{multirow}
\usepackage{caption}
%\usepackage{subcaption}
\usepackage{url}
\usepackage{xspace}
%\usepackage{graphicx}
\usepackage{booktabs}
%\usepackage{mathcomp}
%\usepackage{ifdraft}


\newtheorem{theorem}{Theorem}
\newtheorem{lemma}[theorem]{Lemma}
\newtheorem{proposition}[theorem]{Proposition}
\newtheorem{Observation}[theorem]{Observation}
\newtheorem{corollary}[theorem]{Corollary}
\newtheorem{Claim}[theorem]{Claim}
\newtheorem{Property}[theorem]{Property}
\newtheorem{remark}{Remark}
%\theoremstyle{definition}
\newtheorem{definition}{Definition}
\newtheorem{Problem}{Problem}
\newtheorem{Example}{Example}[section]

\begin{document}
\title{Advanced Techniques for Combinatorial Algorithms}
\author{Gianluca Della Vedova --- Raffaella Rizzi}
\date{2017--18}

\maketitle


\section{Parallel Algorithms}

Given an undirected edge-weighted connected graph $G = (V, E)$, find a minimum-weight
subset $T \subseteq E$ such that $T$ is a tree spanning $V$.


% \section{String Algorithms}

% Recall that the lcp array stores the length of the longest common
% prefix between $w[SA[i]..n]$ and $w[SA[i+1]..n]$, where $SA$ is the suffix
% array. We sketched an $O(n)$ time algorithm for computing all such
% values. Write its pseudocode and prove the correctness.

% \section{Approximation Algorithms}


% Let $G=(V,E)$ be a directed graph (i.e., each arc (v,w) is outgoing from
% v and incoming in w), with $|V|=n$, $|E|=m$.
% Moroever $G$ has no loops (a loop is an arc $(v,v)$).
% Each arc $(v,w)$ has a nonnegative weight $w(v,w)$.

% Let $\pi=<i_1, \ldots, i_n>$ be a permutation of $1, \ldots, n$. Then
% $\pi$ induces the ordering $<v_{i_1}, \ldots, v_{i_n}>$ of the vertices of $G$.
% An arc $(v_i, v_j)$ is consistent with an ordering $\pi$ iff $i$
% precedes $j$ in $\pi$.

% Instance: a directed graph $G$\\
% Feasible solution: an ordering $\pi$ of the vertices of $G$\\
% Objective function: the total weight of the arcs of $G$ that are consistent with
% $\pi$ (to maximize).


% \textbf{Question}.
% Design a $\frac{1}{2}$-approximation algorithm, proving the guaranteed
% approximation ratio.


\end{document}
