\documentclass[12pt,aspectratio=169]{beamer}
\usepackage{fancyvrb}
\RecustomVerbatimCommand{\VerbatimInput}{VerbatimInput}{frame=single,
numbersep=1mm, numbers=left, formatcom=\color{orange}}
%\usepackage{kpfonts}
%\usepackage[bitstream-charter]{mathdesign}
\usepackage[utf8]{inputenc}
\usepackage{pgf}
\usepackage{verbatim}
%\usepackage{fontspec}
\usepackage[ruled,vlined,linesnumbered]{algorithm2e}
\IncMargin{1em}
\usetheme{Madrid}
\setbeamerfont{frametitle}{series=\bfseries}
\usecolortheme[dark]{solarized}
\setbeamertemplate{blocks}[rounded][shadow=false]
\setbeamertemplate{navigation symbols}{}


\author{Gianluca Della Vedova}
\title[Advanced Algorithms]{Advanced Techniques for Combinatorial Algorithms:
External-Memory Algorithms}
\institute[]{Univ. Milano--Bicocca\\
  \texttt{https://gianluca.dellavedova.org}}

\DeclareMathOperator{\poly}{\text{poly}}
\DeclareMathOperator{\polylog}{\text{polylog}}


% If you wish to uncover everything in a step-wise fashion, uncomment
% the following command:
% \beamerdefaultoverlayspecification{<+->}


\begin{document}

\begin{frame}
  \titlepage
\end{frame}



\begin{frame}\frametitle{Memory Hierarchy}
  \begin{itemize}
  \item
    CPU Registers
  \item
    L1 Cache: $32-256$ KBytes, latency $< 10^{-9}$ secs
  \item
    L2 Cache: $1-16$ MBytes, block transfer size $B=32$ Bytes
  \item
    RAM: $1-32$ GBytes,  $B=64$ Bytes
  \item
    Disk: $100$ GBytes - $1$ PBytes,  $B=8$ KBytes, latency $> 10^{-3}$ secs
  \end{itemize}
\end{frame}


\begin{frame}\frametitle{PDM model}
  \begin{itemize}
  \item
    Parallel Disk Model
  \item
    Locality of reference
  \item
    Parallel disk access
  \item
    Disk striping (data across several disks)
  \item
    Count \alert{I/O operations}
  \end{itemize}
\end{frame}

\begin{frame}\frametitle{PDM parameters}
  \begin{itemize}
  \item
    $N$ = problem size (in units of data items)
  \item
$M$ = internal memory size (in units of data items)
  \item
$B$ = block transfer size (in units of data items)
  \item
$D$ = number of independent disk drives;
  \item
$P$ = number of CPUs
\item
  $Q$ = number of queries (for a batched problem);
  \item
$Z$ = answer size (in units of data items).
  \item
    $M < N$ and $1 \le DB \le M/2$
  \item
    $n=N/B$, $m=M/B$, $q=Q/B$, $z=Z/B$
  \end{itemize}
\end{frame}


\begin{frame}\frametitle{Basic operations}
  \begin{itemize}
  \item
    Scan: $\Theta(\frac{N}{DB})=\Theta(\frac{n}{B})$
  \item
    Sort: $\Theta(\frac{N}{DB} \log_{M/B} \frac{N}{B}) = \Theta(\frac{n}{D} \log_{M/B} n)$
  \item
    Search: $\Theta(\log_{DB}N)$
  \item
    Output: $\Theta(\max\{1, \frac{Z}{DB}\}) = \Theta(\max\{1, \frac{z}{D}\})$
  \end{itemize}
\end{frame}


\begin{frame}\frametitle{Disk striping}
  \begin{itemize}
  \item
    I/O only on entire stripes
  \item
    cohesive set of disks
  \item
    $D$ disks as a logical disk with logical block size $DB$
  \end{itemize}
  \begin{block}{Main idea}
    \begin{itemize}
    \item
$1$ disk: each I/O step transmits one block of size $DB$
\item
$D$ disks: each I/O step consists of $D$
simultaneous block transfers of size $B$
each.
\item
  Same number of I/O steps
\end{itemize}
\end{block}
\end{frame}



%   \section{Relations between models}
% \begin{frame}\frametitle{}
%   \begin{center}
%     \Huge
%     Relations between models
%   \end{center}
% \end{frame}

% \begin{frame}\frametitle{Map Reduce and PDM}
%   \begin{block}{Shuffle step}
%   \begin{itemize}
%   \item
%     Matrix: $R$=rows=reduce steps, $M$=columns=keys
%   \item
%     sparse matrix
%   \item
%     Layout depending on maps
%   \item
%     Rearrange into a column-wise ordering
%   \end{itemize}
% \end{block}
% \end{frame}



\section{Algorithms}

\begin{frame}\frametitle{Distribution sort}
  \begin{block}{$S$ buckets}
    \begin{itemize}
    \item
      By choosing $S-1$ pivots
    \item
      needs buckets of similar size, so $O(\log_{S}n)$ recursion layers
    \item
      scan to build the buckets.
%
      When a buffer is full $\Rightarrow$ write it
    \item
      $O(m)$ buckets
    \item
      probabilistic approach to select the pivots
    \end{itemize}
  \end{block}
\end{frame}


  \begin{frame}\frametitle{Multiway Partitioning (PDM)}
  \begin{block}{Multiway Partitioning}
    \begin{itemize}
    \item
      $M=\{ m_{1}, \ldots , m_{d}\}$ ordered set of pivots
    \item
      $S$: unordered set of elements
    \item
      $A_{i}$: $i$-th bucket. $a_{i}\in A_{i}$, $m_{i-1}< a_{i} \le m_{i}$
    \item
      Goal: Compute $A_{i}$s
    \item
      Goal: Compute $|A_{i}|$
    \end{itemize}
  \end{block}
\end{frame}

\begin{frame}\frametitle{Multiway Partitioning (PDM)}
  \begin{algorithm}[H]
    Split $A$ into sets $S_{1}, \ldots , S_{P}$\;
    \ForEach{processor $i$ in parallel}{
      Read the vector of pivots $M$ into the cache\;
      Partition $S_{i}$ into $d$ buckets, $J_{i}$ = number of items in each
      bucket
    }
    Prefix Sums on $\{J_{1}, \ldots , J_{P}\}$ in parallel\;
    \ForEach{processor $i$ in parallel}{
      Write elements $S_{i}$ into memory locations offset appropriately by $J_{i-1}$ and $J_{i}$
    }
    compute $|A_{i}|$s, using the prefix sums stored in $J_{P}$
    \caption{MultiPartition}
  \end{algorithm}
\end{frame}

\begin{frame}\frametitle{PDM references}
  \small
  \begin{itemize}
    \item
Jeffrey Scott Vitter.
Algorithms and Data Structures for External Memory.
Foundations and Trends in Theoretical Computer Science.
Now Publishers, 2008~\url{http://www.ittc.ku.edu/~jsv/Papers/Vit.IO_book.pdf}
\item
L. Arge, M. T. Goodrich, M. Nelson, and N. Sitchinava. Fundamental
parallel algorithms for private-cache chip multiprocessors. In Proceedings
of SPAA ’08, pages 197–206. ACM, 2008.
\item
  Fundamental parallel algorithms for private-cache chip multiprocessors
\url{https://dl.acm.org/citation.cfm?id=1378573}
\end{itemize}
\end{frame}


\end{document}

%%% Local Variables:
%%% mode: latex
%%% TeX-PDF-mode: t
%%% buffer-file-coding-system: utf-8
%%% End:
